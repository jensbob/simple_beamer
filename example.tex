\documentclass{beamer}
\usetheme{blue} %light print

\title{Presentation Title}
\subtitle{Subtitle}
\author{Jens Boberski}
\institute{Universit\"at Duisburg-Essen}
\date{October 2012}

\begin{document}
\titleframe
\outlineframe
 \section{Boxes}
\begin{frame}
  \frametitle{A regular box with text}
  
  \begin{cblock}{6cm}
    Equation of motion:
    \begin{equation*}
      \partial_t p_\alpha + \partial_\beta (v_\beta p_\alpha) = \partial_\beta \sigma_{\alpha \beta} + b_\alpha
    \end{equation*}
    \vspace{-2em}
  \end{cblock}
This is done with the cblock environment with the horizontal size as an argument.
  \textref{and some credits with textref}   
\end{frame}
\section{Shaded Text}
\begin{frame}
    \frametitle{Custom shaded text}
    \textsh{\Huge{This is very important}}

    Just use textsh
\end{frame}


\section{Bullet Points}
\begin{frame}
  \frametitle{If you have to use them}
  \begin{itemize}
    \item This is one item
    \item Here is another one
      \begin{itemize}
        \item a subitem
      \end{itemize}
  \end{itemize}
\end{frame}


\end{document}
